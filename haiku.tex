\title{\textbf{Haiku}\\OS Report Part I}
\author{COMP 3000\\ \\Troy Hildebrandt -- 100622385\\Nima Hoda -- 100308135}
\date{\today}

\documentclass{article}

\usepackage{cite}
\usepackage{url}
\usepackage{graphicx}
\usepackage{subfig}

% Page setup
% Shamelessly stolen from Pat Morin's Open Data Structures
\setlength{\textheight}{8.5in}
\setlength{\textwidth}{6in}
\setlength{\topmargin}{-0.375in}
\setlength{\oddsidemargin}{.25in}
\setlength{\evensidemargin}{.25in}
\setlength{\headheight}{0.200in}
\setlength{\headsep}{0.4in}
\setlength{\footskip}{0.500in}
\setlength{\parskip}{1.5ex}
\setlength{\parindent}{1.25cm}
%\flushbottom

\begin{document}
\maketitle

\section{Background}

Haiku is a portable \cite{HaikuFaq} free and open source
\cite{HaikuDevFaq} operating system for the 32-bit x86 architecture
\cite{HaikuFaq} targeting personal computing \cite{HaikuAbout}.  It
was originally named OpenBeOS after BeOS \cite{HaikuWiki}, the
short-lived proprietary operating system developed by Be,
Inc. throughout the 1990s, which focused on digital media work
\cite{BeosWiki}.

BeOS boasted an object-oriented C++ API, a fully integrated graphical
user environment, a modern multiprocessing, preemptively multitasking
kernel, a journaling file system with a relational database-like
metadata facility \cite{BFSWiki} and included partial POSIX
compatibility.  However, despite having a devoted niche user base, it
could not yield a profit for Be, Inc. and, in 2001, was acquired by
Palm, Inc., ending further BeOS development \cite{BeosWiki}.

Disapointed with the loss, a group of BeOS enthusiasts began an
effort to rewrite BeOS as a free and open source project
\cite{BeosWiki, HaikuHistoryWiki}.  In 2004 the open source effort was
renamed Haiku, after the three line Japanese form of poetry which
appeared on occasion in BeOS error messages \cite{HaikuFaq,
  HaikuHistoryWiki}.

Haiku takes its inspiration from BeOS \cite{HaikuAbout} and aims to
recreate both the BeOS technologies and end user experience
\cite{HaikuFaq}.  Haiku uses the same window decoration style and
colour as BeOS \cite{HaikuWiki, BeosWiki} and aims to create a
unified, cohesive user interface \cite{HaikuAbout, HaikuHIG,
  HaikuIcon} reflecting the ``core qualities'' of ``elegance'' and
``simplicity'' found in BeOS \cite{HaikuFaq}.  Haiku also directly
incorporates two core BeOS user interface elements, the Tracker, a
file manager, and the Deskbar, a taskbar, which were open sourced by
Be, Inc. in 2001 \cite{HaikuFaq}.  Haiku also reimplements the
object-oriented C++ BeOS API\cite{HaikuWiki} as well as the Be File
System with, its advanced meta-data capabilities \cite{BFSWiki}.

Haiku is targetted towards the personal computer user
\cite{HaikuAbout}.  As such, it includes a fully integrated GUI
environment, rejecting the stacked nature of the X Window System on
UNIX-like operating systems \cite{HaikuFaq}.  The personal computing
focus is also reflected in Haiku's current lack of support for
multi-user operation, though are plans for such support in future
releases \cite{HaikuFuture}.  Haiku also aims to present a unified
human interface, both in terms of the operating system base as well as
in applications developed for Haiku.  This is reflected in the
development and maintenance of comprehensive Haiku human interface and
icon guidelines \cite{HaikuHIG, HaikuIcon}.

Haiku currently only supports the 32-bit x86 architecture, though
ports to other platforms, including x86-64, are being developed and
may be supported in the future \cite{HaikuFaq}.  On 32-bit x86, Haiku
aims for full source and binary compatibility with 32-bit x86 BeOS
applications.  Several major BeOS R5 applications already run
successfully on Haiku, including Opera, Firefox and Quake III
\cite{HaikuWiki}.  POSIX compatibility is also a Haiku goal
\cite{HaikuFuture, HaikuIncContracts}, as are the development of a
BSD-style ports collection \cite{HaikuPorts}, which would simplify the
building of 3rd party open-source software and a full package
management system to manage installation and dependency resolution of
3rd party software \cite{HaikuR1A3Notes}.

Haiku development is centred around the Haiku Project, an
international community of volunteers \cite{HaikuAbout}.  The project
maintains mailing lists, forums, IRC channels, \cite{HaikuComm} source
code repositories \cite{HaikuGetSvn} and several websites, including
the following:
\begin{itemize}
  \item \url{http://www.haiku-os.org/} -- the main project website
  \item \url{http://api.haiku-os.org/} -- containing Haiku API documentation
  \item \url{http://dev.haiku-os.org/} -- the Haiku project management
    system
  \item \url{http://www.haiku-files.org/} -- an archive of nightly
    builds for Haiku
  \item \url{http://ports.haiku-files.org/} -- the Haiku ports collection
    project management system
\end{itemize}

There are 81 developers with commit access to the Haiku source code
repository, with the top contributors ranking in the thousands of
commits over many years \cite{HaikuContrib}.  The source code is
revision controlled by an SVN repository and managed by Trac
\cite{HaikuDevStart}.  Write access to the repository is granted to
contributors on the basis of past contributions \cite{HaikuDevStart}.

In addition to voluntary contributions, the Haiku Project has also
participated in the Google Summer of Code (GSOC)---a program that
provides stipends to students who contribute to open-source software
\cite{GSOCWiki}---every year since 2007 \cite{HaikuGSOC}.  In 2010,
Haiku took contributions from seven students through GSOC
\cite{HaikuGSOC2010}.  Contributors are also sometimes contracted to
work full-time on Haiku, allowing them to dedicate large continuous
blocks of time to Haiku development \cite{HaikuIncContracts}.
Contract lengths have mostly been limited to several weeks but also
include a recently awarded six month contract
\cite{HaikuLongContract}.

The Haiku Project is supported financially by Haiku, Inc., a 501(c)(3)
charitable non-profit corporation founded in 2003 by Michael Phipps in
New York State's Division of Corporations \cite{HaikuIncAbout,
  HaikuInc}.  Haiku, Inc., funded through donations, has a 2011
operating budget of \$25,500 \cite{HaikuIncDocs}.  Donations to Haiku,
Inc. come almost entirely from private individuals, either directly to
Haiku, Inc. or through ``bounties'' and ``Thank You Awards'' from
HaikuWare \cite{HaikuIncDonors, HaikuWareBounties}, a repository of
pre-built software for Haiku as well as a hub for Haiku end-users
\cite{HaikuWareAbout}.

Haiku Release 1 Alpha 3 was published on June 20, 2011
\cite{HaikuRelease}.  It is available for free download from various
FTP and HTTP mirrors in the following formats \cite{HaikuGet}:
\begin{itemize}
\item ``Anyboot'' images -- which may be written to and booted from
  USB flash drives, hard disk drives or CD/DVD media
\item ISO images -- which may be written to CD media
\item VMDK images -- which are virtual machine disk files compatible
  with VMWare products as well as QEMU and VirtualBox \cite{VMDKWiki}
\end{itemize}
Any of the image formats may be used live, to try Haiku without
installing it, or to install Haiku to another disk volume
\cite{HaikuGet}.  The Haiku alpha release is also available for order
on a ``commemorative'' CD from Haiku, Inc. for a minimum donation of
\$10 per disc \cite{HaikuIncOrder}.  Source code for this release is
also available for download in compressed archived form
\cite{HaikuR1A3Src}.

The latest Haiku source code can be obtained directly from the source
code repository \cite{HaikuGetSvn} and nightly snapshot builds are
available as downloadable images from the Haiku Files archive
\cite{HaikuFiles}.

The Haiku release (anyboot) is approximately 250MB in size compressed
\cite{HaikuGet} and 700MB uncompressed.  The installed system also
takes up about 700MB of disk space, not including the size of a swap
file, which is 2GB by default.  The source distribution is split into
two files.  One containing the Haiku build tools, comprised almost
entirely of customized third party software (e.g. GCC, binutils) and
having an uncompressed size of about 530MB, and the other containing
the Haiku system sources proper and having approximately the same
uncompressed size \cite{HaikuR1A3Src}.  The Haiku system sources
contain about 5.5 million lines of C and C++ code and headers, not
including blank or comment lines, as reported by CLOC \cite{Cloc}.

\section{Installation Procedure}

See figure \ref{fig:foo}.

\begin{figure}[h]
\centering
\includegraphics{figs/somefig.mps}
\caption{Foo.}
\label{fig:foo}
\end{figure}

For the purposes of testing, installation was performed on an Oracle 
VirtualBox setup, in this specific case using Windows 7 64bit host as
a host OS. The OS was set to Other/Unknown, and 512MB of memory was 
allotted for Haiku to use. The default values of 2GB for boot disk size,
and another 2GB for Haiku’s usable partition was selected as the starting 
amount, with “Dynamically expanding storage” selected just in case.

Installation began by selecting the Haiku disc ISO to be used for 
the secondary IDE device in the Storage Settings for my Haiku guest, and 
a first boot was initiated. <insert install1.png> After a simple language 
select, a quick and painless format of my 2GB partition to the Be File 
system, my partition was ready to be selected from a drop down list and 
the install could begin. This process could simply not have been more 
painless, and there were absolutely no surprises or hurdles to overcome 
during the process. <insert install2.png> After the progress bar reached 
the end, I chose the “Quit”option and the system rebooted to  be used 
for the first time.

\section{Startup}

The first start-up presents the user with a screen showing the Haiku logo, 
and a series of logos that go from monochrome to full color in sequence 
to indicate progress during boot-up.  <Insert startup1.png> Given this 
execution combined with the artistic styling of the logos, this is very 
reminiscent of Maxis’ The Sims series of games.
 
After what is a noticeably rapid boot process, users familiar with Haiku’s 
inspiration and spiritual predecessor, BeOS, will be instantly greeted with 
both familiar desktop and icon styling along with the Tracker, one of very 
few things that made a direct translation from BeOS to Haiku as part of 
being open-sourced in 2001.\cite{HaikuFaq}

There is also a very profound lack of any further setup involved, and Haiku is, 
for all intents and purposes, ready for use. The time taken from start to finish 
in setting up an installation of Haiku under VirtualBox is roughly five minutes.

\section{Basic Operation}

Haiku comes pre-installed with a host of applications that the average user may 
use, the most important of which is likely its pre-installed browser, titled 
WebPositive. Haiku is aimed at “personal computing”, and given this vague 
definition, seems to fulfill the needs of an average light user “out of the 
box.” The web browser will be the obvious first stop for most users looking 
to get more software and make their Haiku experience a more productive one.
	
This is where the first very minor problem was encountered, and it 
was a network related issue. Haiku refused to recognize the default network 
card that VirtualBox had chosen to virtualize, and so a quick shutdown and 
modification of this setting to a card that Haiku recognizes solved the problem 
immediately.
	
A user looking to expand their selection of installed software will 
likely end up at the site Haikuware. (www.haikuware.com) Considering my astounding 
disinterest in word-processing or productivity programs, I decided to hunt 
down a Haiku-compatible version of a classic favourite of mine, id Software’s 
Quake II. <insert using1.jpg> Interestingly enough, the claim from the 
download page at Haikuware for Quake II claims it is the same version that 
used to be found on BeDepot.
	
Following my download's completion, I noticed it was of the file format .
pkg. A double click revealed a very simple auto-install package that I allowed 
to install to its default directory, /boot/apps. The file system in BeOS/Haiku 
is very similar to that of a Unix OS, despite Haiku not being a Unix based 
operating system. Haiku also comes with a Bash terminal, with the usual commands 
like ls, chmod, pwd, and many others. <insert using2.png> The mounting of a 
CD is automatic, and very easily accessible once inserted, either from a link 
on the desktop, or from the root directory, based on what the name of the CD 
is. In this case, I could access the files on the Quake II CD from /Quake2, 
and the files from the installation of BeOS Quake II at /boot/apps/Quake2.
	
After copying the contents of a full Windows installation of Quake 
II into Haiku via USB drive (which again was instantly recognized and mounted), 
I started it up. The first noticeable problems were both the obviously broken 
and ear-piercing sound which caused me to mute my PC, and the excruciatingly 
slow Mesa software OpenGL implementation that the game defaulted to. This is 
a result of no Guest Additions and no virtualized OpenGL 3D-acceleration, 
and as a result, I switched to pure software rendering mode which ended up 
working out nicely. <insert using3.png> However, attempting to start a game 
causes a crash, and while playing the beginning demo the game will occasionally 
crash as well. I’m not sure if this is a result of it being a BeOS executable, 
the fact that I’m running it in VirtualBox, or just that the BeOS/Haiku executable 
of Quake II isn’t quite ready for prime time yet. I was nevertheless still impressed 
to see it running in any capacity in Haiku.
	
Most people will at some point want to watch a video on Youtube, or at 
least some Flash-based video website. If Haiku was going to be of any use to the 
average personal computer user, I was going to need to determine whether or not 
Youtube was a viable option in Haiku. I set about trying to download a new browser 
from Haikuware to spruce up my browsing experience, and to answer the question 
“how simple is the installation of new software like a web browser?”  I soon 
discovered the answer is “not very.”

There is a small selection of browsers available for Haiku directly from Haikuware, 
but they’re mostly old versions of browsers (i.e. Firefox 2.0 since FF3 still isn’t 
fully ported, and older versions of Opera) that still don’t have proper plugin 
support for things like Flash. Since Flash is closed-course, open-source versions 
of it like Gnash are made, but are still largely unsupported and not optimal. 
This makes something like watching Youtube a more massive undertaking than most 
basic users will be capable of, since getting Flash support in a web browser in 
Haiku is a task in itself.  This speaks poorly to Haiku’s goal to be for “personal computing”, 
considering how many more popular operating systems provide this basic functionality.

\section{Usage Evaluation}

While a seemingly faithful recreation of the original BeOS and an impressive piece of work in 
its own right, Haiku falls short of its goals to be for “personal computing” by simply not 
having enough software support and lacking in the excellent package distribution models 
that many other more popular operating systems provide. An interesting OS, and considering 
it’s open source model could technically be made to do anything if you felt so inclined, 
but the types of users who would delve into source code to coax Haiku into being more 
useful aren’t the sort of people I feel the Haiku developers are speaking to when they 
describe it vaguely as being for “personal computing.”

\bibliography{haiku}{}
\bibliographystyle{plain}

\end{document}
